\documentclass[12pt,a4paper,english]{article}
\usepackage [utf8]{inputenc}
\usepackage [english]{babel}
\usepackage [T1]{fontenc}
\usepackage {amsmath}
\usepackage{amsfonts}
\usepackage{amssymb}
\usepackage [top=3.4cm, bottom=2cm, left=3cm, right=3cm] {geometry}
\usepackage{setspace}
\usepackage{titlesec}
\usepackage{pgfkeys}
\titleformat{\section}[block]{\Large\bfseries\filcenter}{}{1em}{}
\titleformat{\subsection}[hang]{\bfseries\filcenter}{}{1em}{}
\setcounter{secnumdepth}{0}
\usepackage{pgfgantt}
\setcounter{tocdepth}{3}
\usepackage{graphicx}
\usepackage{geometry}
\usetikzlibrary{positioning,shapes, shadows, arrows}
\usepackage{tikz}
\usepackage{url}
\linespread{1.2}
\usepackage{fancyhdr}
\pagestyle{fancyplain}
\fancyhead[L]{Paladim Assignment}
\fancyhead[C]{}
\fancyhead[R]{December 20 2013}
\usepackage{listings}
\lstset{moredelim=[s][\color{gray}]{(*}{*)}}
\lstset{moredelim=[s][\color{green!50!brown}]{"}{"}}
\lstset{moredelim=[s][\color{gray}]{/*}{*/}}
\lstset{emph={char,int,bool,string,array,Bool,Int,Char,Array,String},emphstyle=\color{yellow!50!brown},
	  emph={[2]NONE,SOME},emphstyle={[2]\color{orange}},
	  emph={[3]let,fun,map,val,in,case,of,end,and,token,type,nonassoc,right,left,start,handle,if,then,else,andalso,orelse},emphstyle={[3]\color{brown}},
	  emph={[4]call_args,hitler,updateOuterVtable,lenin,evalExp,
	  evalOr,evalBinop,evalNot,print, compileExp, pp_exp, typeOfExp, posOfExp, Times, Div, Or, Not,pp_type,typeCheckExp,Mips},emphstyle={[4]\color{blue}},
	  emph={[5]},emphstyle={[5]\color{orange!50!red}}}


\begin{document}

\begin{titlepage}
    \vspace*{\fill}
    \begin{center}
      {\Huge Paladim Compiler}\\[0.7cm]
      {\large Alexander Worm Olsen - bdj816}\\[0.4cm]
      {\large Chi Dan Pham - vqr853}\\[0.4cm]
      {\large Troels Thompsen - qvw203}\\[0.4cm]
      {\small Group Project}\\[0.3cm]
      {\small December 20 2013}\\[0.3cm] 
      {\small Departmen of Computer Science}\\
      {\small University of Copenhagen}
    \end{center}
    \vspace*{\fill}
\end{titlepage}

\tableofcontents
\newpage
\section{Introduction}

\begin{lstlisting}
in  ( case (rtp, res) of
      (NONE , _) =>
           let 
             fun call_args [] [] = NONE
               | call_args (x::xs) (y::ys) =
                 let 
                   val hitler = updateOuterVtable vtab new_vtab (x, y)
                 in
                   call_args xs ys
                 end
               | call_args _ _ = raise Error
				       ("Number of functions " ^ 
			                "args does not match " ^ 
				        "declaration, at ", pdcl)
                 in 
                   call_args aexps fargs  
                 end
\end{lstlisting}

\begin{lstlisting}
and updateOuterVtable vtabOuter vtabInner 
                     (TpAbSyn.LValue (lval1, pos1), 
	              TpAbSyn.Dec ((id2,tp), pos2)) = 
    let 
      val lenin = ( case lval1 of 
                    Var(id,tp) => id
                  | Index((id,tp),e) => id)
    in
      case (SymTab.lookup lenin vtabOuter, 
            SymTab.lookup id2 vtabInner) of
      (SOME x, SOME y) => x := !y
      | _              => raise Error("Procedure argument " ^ 
				      "not in caller", pos1)
    end
|   updateOuterVtable _ _ _ = raise Fail("Internal type error, " ^ 
                                         "see updateOuterVtable. \n")
\end{lstlisting}
\newpage
\begin{lstlisting}
  | evalExp ( Div(e1, e2, pos), vtab, ftab ) =
        let val res1   = evalExp(e1, vtab, ftab)
            val res2   = evalExp(e2, vtab, ftab)
        in  evalBinop(op div, res1, res2, pos)
        end

  | evalExp ( Times(e1, e2, pos), vtab, ftab ) =
        let val res1   = evalExp(e1, vtab, ftab)
            val res2   = evalExp(e2, vtab, ftab)
        in  evalBinop(op *, res1, res2, pos)
        end

  | evalExp ( Or(e1, e2, pos), vtab, ftab ) =
        let val r1 = evalExp(e1, vtab, ftab)
            val r2 = evalExp(e2, vtab, ftab)
  	in  evalOr(r1, r2, pos)
  	end
    
  | evalExp ( Not(e1, pos), vtab, ftab ) =
        let val r1 = evalExp(e1, vtab, ftab)
        in  evalNot(r1, pos)
        end

fun evalOr  (BVal (Log b1), BVal (Log b2), pos) = 
	     BVal (Log (b1 orelse b2))
  | evalOr  (v1, v2, pos) =
        raise Error( "Or: argument types do not match. Arg1: " ^
                      pp_val v1 ^ ", arg2: " ^ pp_val v2, pos )
                      
fun evalNot  (BVal (Log b1), pos) = BVal (Log (not b1))
  | evalNot  (v1, pos) =
        raise Error( "Not: argument types do not match. Arg1: " ^
                      pp_val v1, pos )
\end{lstlisting}

\newpage
\begin{lstlisting}
%{ 
   
%}

/* Token type definitions (will often be used in the Lexer)
 * Tokens use position attribute for demonstration 
 * (see below for Lexer)
 * As mentioned, the SML code above ends up after
 * this data declaration,
 * so we cannot use any types defined above
 * at this point of the file.
 * tokens needs to be of sml types
 * through the rules the should be converted to an absyn syntaxtree */

%token <AbSyn.Pos> TProgram TFunction TProcedure TVar TBegin TEnd 
		   TIf TThen TElse TWhile TDo TReturn TArray TOf 
		   TInt TBool TChar TAnd TOr TNot TAssign TPlus TMinus 
		   TTimes TSlash TEq TLess TLParen TRParen TLBracket 
		   TRBracket TLCurly TRCurly TComma TSemi TColon TEOF
%token <bool*AbSyn.Pos> TBLit
%token <int*AbSyn.Pos> TNLit
%token <char*AbSyn.Pos> TCLit
%token <string*AbSyn.Pos> TSLit TId

%nonassoc LowPrec
%nonassoc TElse
%right TOr
%right TAnd
%nonassoc TNot
%nonassoc TEq TLess
%left TPlus TMinus
%left TTimes TSlash

/* start symbol */
%start Program

/* types returned by rules below */
%type <AbSyn.Prog> Program
%type <AbSyn.Prog> Prog
%type <AbSyn.Prog> FunDecs
%type <AbSyn.FunDec> FunDec
%type <AbSyn.StmtBlock> Block 
%type <AbSyn.Stmt list> SBlock
%type <AbSyn.Stmt list> StmtSeq
%type <AbSyn.Stmt> Stmt
%type <AbSyn.LVAL*AbSyn.Pos> LVal
%type <AbSyn.Exp option> Ret
%type <AbSyn.Exp> Exp
%type <AbSyn.Dec list> PDecl Params
%type <AbSyn.Dec> Dec
%type <AbSyn.Dec list> Decs
%type <AbSyn.Type> Type
%type <AbSyn.Exp list> CallParams Exps

%%

/* rules - a separate start rule is added automatically */

/* PROGRAM STRUCTURE*/
Program: Prog TEOF                 { $1 } 
;

Prog : 
    TProgram TId TSemi FunDecs     { $4 }
;

FunDecs :
    FunDecs FunDec                 { $1 @ [$2] }
  | FunDec                         { [$1] }
;

FunDec :
    TFunction TId TLParen PDecl TRParen TColon Type Block TSemi
                                   { AbSyn.Func($7, #1 $2, $4, $8, $1) }
  | TProcedure TId TLParen PDecl TRParen Block TSemi
                                   { AbSyn.Proc(#1 $2, $4, $6, $1) }
;

Block :
    SBlock                         { AbSyn.Block ([], $1) }
  | TVar Decs SBlock               { AbSyn.Block ($2, $3) }
;

SBlock :
    TBegin StmtSeq TSemi TEnd      { $2 }
  | Stmt                           { [$1] }
;

StmtSeq :
    StmtSeq TSemi Stmt             { $1 @ [$3] }
  | Stmt                           { [$1] }
;

/* STATEMENTS */
Stmt :
    TId TLParen CallParams TRParen { AbSyn.ProcCall (#1 $1, $3, #2 $1) }
  | TIf Exp TThen Block TElse Block{ AbSyn.IfThEl ($2, $4, $6, $1) }
  | TIf Exp TThen Block %prec LowPrec 
			           { AbSyn.IfThEl ($2, $4, 
						   AbSyn.Block([],[]), 
					    	   $1)}
  /* prec gives precedence as LowPrec */
  | TWhile Exp TDo Block           { AbSyn.While ($2, $4, $1) }
  | TReturn Ret                    { AbSyn.Return ($2, $1) }
  | LVal TAssign Exp               { AbSyn.Assign ($1, $3, $2) }
;

/* L-VALUES AND EXPRESSIONS */
LVal :
    TId                            { AbSyn.Var (#1 $1) }
  | TId TLBracket Exps TRBracket   { AbSyn.Index (#1 $1, $3) }
;

Ret :
    Exp                            { SOME $1 }
  |                                { NONE }
;
\end{lstlisting}
\newpage
\begin{lstlisting}
Exp :
    TNLit                          { AbSyn.Literal (AbSyn.BVal(
				     AbSyn.Num(#1 $1)), #2 $1) }
  | TBLit                          { AbSyn.Literal (AbSyn.BVal(
				     AbSyn.Log(#1 $1)), #2 $1) }
  | TCLit                          { AbSyn.Literal (AbSyn.BVal(
				     AbSyn.Chr(#1 $1)), #2 $1) } 
  | TSLit                          { AbSyn.StrLit (#1 $1, #2 $1) }
  | TLCurly Exps TRCurly           { AbSyn.ArrLit ($2,$1) }
  | LVal                           { AbSyn.LValue (#1 $1, #2 $1) }
  | TNot Exp                       { AbSyn.Not ($2, $1) }
  | Exp TPlus Exp                  { AbSyn.Plus ($1, $3, $2) }
  | Exp TMinus Exp                 { AbSyn.Minus ($1, $3, $2) }
  | Exp TTimes Exp                 { AbSyn.Times ($1, $3, $2) }
  | Exp TSlash Exp                 { AbSyn.Div ($1, $3, $2) }
  | Exp TEq Exp                    { AbSyn.Equal ($1, $3, $2) }
  | Exp TLess Exp                  { AbSyn.Less ($1, $3, $2) }
  | Exp TAnd Exp                   { AbSyn.And ($1, $3, $2) }
  | Exp TOr Exp                    { AbSyn.Or ($1, $3, $2) }
  | TLParen Exp TRParen            { $2 }
  | TId TLParen CallParams TRParen { AbSyn.FunApp (#1 $1, $3, # 2$1) }
;

/* VARIABLE AND PARAMETER DECLARATIONS, TYPES */
PDecl : 
    Params                         { $1 }
  |                                { [] }
;

Params :
    Params TSemi Dec               { $1 @ [$3] }
  | Dec                            { [$1] }
;

Dec :
    TId TColon Type                { AbSyn.Dec (#1 $1, $3, #2 $1) }
;
\end{lstlisting}
\newpage
\begin{lstlisting}
Decs :
    Decs Dec TSemi                 { $1 @ [$2] }
  | Dec TSemi                      { [$1] }
;

Type :
    TInt                           { AbSyn.Int ($1) }
  | TChar                          { AbSyn.Char ($1) }
  | TBool                          { AbSyn.Bool ($1) }
  | TArray TOf Type                { AbSyn.Array ($3,$1) }
;

/* FUNCTION AND PROCEDURE PARAMETERS AND INDEX LISTS */
CallParams :
    Exps                           { $1 }
  |                                { [] }
;

Exps :
    Exp TComma Exps                { $1 :: $3 }
  | Exp                            { [$1] }

%%

(* SML trailer *)
\end{lstlisting}
\newpage
\begin{lstlisting}
fun compile arg path =
  let
    val inpath = path
    val outpath= Path.base path ^ ".asm"
    val lexbuf = createLexerStream (BasicIO.open_in inpath)
  in
    let
      (*val pgm = LL1Parser.parse Lexer.Token lexbuf*)
      (* COMMENT LINE ABOVE AND UNCOMMENT  *)
      (* THE LINE BELOW TO USE YOUR PARSER *)
      val pgm = Parser.Program Lexer.Token lexbuf
    in case arg of
      "-ti" => typedInterpret (typeCheck pgm)
    | "-c"  => compileNormal pgm outpath
    | other => print ("'" ^ other ^ "': Unknown mode of operation.\n")
    end
    handle
      Parsing.yyexit ob => errorMsg "Parser-exit\n"
    | Parsing.ParseError ob =>
       (* errorMsgAt "Parsing error" (Lexer.getPos lexbuf) *)
       (* COMMENT LINE ABOVE AND UNCOMMENT  *)
       (* THE LINE BELOW TO USE YOUR PARSER *)
         errorMsgAt "Parsing error" (Lexer.getPos lexbuf)

   (* | Parser.ParseError s =>
         errorMsgAt ("Parse error: " ^ s) (Lexer.getPos lexbuf) *)
\end{lstlisting}

\begin{lstlisting}
| compileExp( vtable, Times (e1, e2, _), place ) =
    let val t1 = "times1_" ^ newName()
        val c1 = compileExp(vtable, e1, t1)
        val t2 = "times2_" ^ newName()
        val c2 = compileExp(vtable, e2, t2)
    in c1 @ c2 @ [Mips.MUL (place,t1,t2)]
    end

| compileExp( vtable, Div (e1, e2, _), place ) =
    let val t1 = "div1_" ^ newName()
        val c1 = compileExp(vtable, e1, t1)
        val t2 = "div2_" ^ newName()
        val c2 = compileExp(vtable, e2, t2)
    in c1 @ c2 @ [Mips.DIV (place,t1,t2)]
    end
\end{lstlisting}

\begin{lstlisting}
| compileExp( vtable, Or(e1, e2, _), place ) =
    let val t1 = "or1_" ^ newName()
        val c1 = compileExp(vtable, e1, t1)
        val t2 = "or2_" ^ newName()
        val c2 = compileExp(vtable, e2, t2)
        val lA = "_or_" ^ newName()
    in 
        c1 @ c2 @ [Mips.OR (place,t1,t2)]
    end
        
| compileExp( vtable, Not(e1, _), place ) =
    let val t1 = "not1_" ^ newName()
        val c1 = compileExp(vtable, e1, t1)
        val lA = "_not_" ^ newName()
    in 
        c1 @ [Mips.XORI (place,t1,"1")]
    end
\end{lstlisting}

\begin{lstlisting}
| Times   of Exp * Exp         * Pos      (* e.g., x * 3 *)
| Div     of Exp * Exp         * Pos      (* e.g., x / 3 *) 
| Or      of Exp * Exp         * Pos      (* e.g., (x=5) or y *)
| Not     of Exp               * Pos      (* e.g., not (x>3) *)

| pp_exp (Or    (e1, e2, _))    = "( " ^ pp_exp e1 ^ " | " ^ pp_exp e2 ^ " )"
| pp_exp (Not   (e1,     _))    = "( not " ^ pp_exp e1 ^ " )"
| pp_exp (Times (e1, e2, _))    = "( " ^ pp_exp e1 ^ " * " ^ pp_exp e2 ^ " )"
| pp_exp (Div   (e1, e2, _))    = "( " ^ pp_exp e1 ^ " / " ^ pp_exp e2 ^ " )"

| typeOfExp ( Or     (_,_,_) ) = BType Bool
| typeOfExp ( Not    (_,  _) ) = BType Bool
| typeOfExp ( Times  (a,b,_) ) = typeOfExp a
| typeOfExp ( Div    (a,b,_) ) = typeOfExp a

| posOfExp  ( Or     (_,_,p) ) = p
| posOfExp  ( Not    (_,  p) ) = p
| posOfExp  ( Times  (_,_,p) ) = p
| posOfExp  ( Div    (_,_,p) ) = p

\end{lstlisting}

\begin{lstlisting}
(* Must be modified to complete task 3 *)
| typeCheckExp( vtab, AbSyn.Div (e1, e2, pos), _ ) =
   let val e1_new = typeCheckExp(vtab, e1, KnownType (BType Int) )
        val e2_new = typeCheckExp(vtab, e2, KnownType (BType Int) )
        val (tp1, tp2) = (typeOfExp e1_new, typeOfExp e2_new)
   in  if  typesEqual(BType Int, tp1) andalso 
            typesEqual(BType Int, tp2)
        then Div(e1_new, e2_new, pos)
        else raise Error("in type check minus exp, one argument " ^
			 "is not of int type "^ pp_type tp1^
			 " and "^pp_type tp2^" at ", pos)
   end
          
(* Must be modified to complete task 3 *)
| typeCheckExp ( vtab, AbSyn.Equal(e1, e2, pos), _ ) =
  let val e1_new = typeCheckExp(vtab, e1, UnknownType)
      val e2_new = typeCheckExp(vtab, e2, 
		   		KnownType (typeOfExp e1_new) )
      val (tp1, tp2) = (typeOfExp e1_new, typeOfExp e2_new)
      (* check that tp1 is not an array type *)
      val () = case tp1 of
               Array _ => raise Error("in type check equal, " ^
				      "first expression " ^
				      pp_exp e1_new ^
                                      "is an array (of type) " ^
				      pp_type tp1 ^ " at ", pos)
                   | _ => ()
  in  if  typesEqual(tp1, tp2)
      then Equal(e1_new, e2_new, pos)
      else raise Error("in type check equal exp, " ^
		       " argument types do not match " ^
                       pp_type tp1^" <> "^pp_type tp2^" at ", pos)
  end
\end{lstlisting}
\newpage
\begin{lstlisting}
(* Must be modified to complete task 3 *)
| typeCheckExp ( vtab, AbSyn.Less (e1, e2, pos), _ ) =
    let val e1_new = typeCheckExp(vtab, e1, KnownType (BType Int) )
        val e2_new = typeCheckExp(vtab, e2, KnownType (BType Int) )
        val (tp1, tp2) = (typeOfExp e1_new, typeOfExp e2_new)
        (* check that tp1 is not an array type *)
        val () = case tp1 of
                   Array _ => raise Error("in type check less, " ^
					  "first expression "^ 
					  pp_exp e1_new ^
                                          "is an array (of type) " ^
					  pp_type tp1^" at ", pos)
                       | _ => ()
    in  if  typesEqual(tp1, tp2)
        then Less(e1_new, e2_new, pos)
        else raise Error("in type check less exp, argument types " ^
			 " do not match "^
                         pp_type tp1^" <> "^pp_type tp2^" at ", pos)
    end

\end{lstlisting}
\newpage
\begin{lstlisting}
(* Must be modified to complete task 3 *)
| typeCheckExp ( vtab, AbSyn.Or (e1, e2, pos), _ ) =
    let val e1_new = typeCheckExp(vtab, e1, KnownType (BType Bool) )
        val e2_new = typeCheckExp(vtab, e2, KnownType (BType Bool) )
        val (tp1, tp2) = (typeOfExp e1_new, typeOfExp e2_new)
    in  if  typesEqual(BType Bool, tp1) andalso 
	    typesEqual(BType Bool, tp2)
        then Or(e1_new, e2_new, pos)
        else raise Error("in type check and exp, one argument is " ^ 
			 "not of bool type "^
                           pp_type tp1^" and "^pp_type tp2^" at ", pos)
    end    

(* Must be modified to complete task 3 *)
| typeCheckExp ( vtab, AbSyn.Not (e1, pos), _ ) =
    let val e1_new = typeCheckExp(vtab, e1, KnownType (BType Bool) )
        val (tp1) = (typeOfExp e1_new)
    in  if  typesEqual(BType Bool, tp1)
        then Not(e1_new, pos)
        else raise Error("in type check and exp, one argument is not " ^
			 "of bool type " ^
                            pp_type tp1^" at ", pos)
    end   
\end{lstlisting}
\newpage
\begin{lstlisting}
(* function call to `new' uses expected type to infer
   the to-be-read result *)
| typeCheckExp ( vtab, AbSyn.FunApp ("new", args, pos), etp ) =
    ( case expectedBasicType etp of
        SOME btp => 
          let 
            val typedargs = map (fn n => 
		      typeCheckExp(vtab, n,KnownType (BType Int))) args
            val types = map typeOfExp typedargs
            val rtp = Array ( length args, btp)
          in
            if List.all (fn n => typesEqual(BType Int, n)) types
            then 
              FunApp(("new", (types, SOME rtp)), typedargs,pos)
            else 
              raise Error("declared array dimensions are not " ^
			  "integers, at ", pos)
          end
\end{lstlisting}

\begin{lstlisting}
| typeCheckExp( vtab, AbSyn.LValue( AbSyn.Index(id, inds), pos ), _ ) =
    let
       val indices = length inds
       val newinds = map (fn n=>
			typeCheckExp(vtab,n,KnownType (BType Int))) inds
       val correctinds = List.all (fn n => 
			typesEqual(BType Int, typeOfExp n)) newinds
    in
       case SymTab.lookup id vtab of
          SOME (Array(r,t)) => 
             if indices > 0 andalso r = length inds andalso correctinds 
             then
               LValue(Index((id,Array(r,t)),newinds),pos)
             else
               raise Error("ill-formed array indexing at ",pos)
        | _ => raise Error("in type check variable, var " 
			   ^id^"not in VTab, at",pos)
    end
\end{lstlisting}
\newpage
\begin{lstlisting}
| compileLVal( vtab : VTab, Index ((n,t),inds) : LVAL, pos : Pos ) =
    ( case SymTab.lookup n vtab of
        SOME mem => 
          let
            val rank = length inds
            val strides = 4 * (rank - 1)
                
            (* Variables for generated MIPS code *)
            val arrptr = "_arrptr_" ^ newName() 
		(* pointer to a *)
            val ind_reg = "_indx_" ^ newName() 
		(* current index i_k *)
            val dim_reg = "_dimx_" ^ newName() 
		(* current dimension d_k *)
            val flat = "_flatIndx_" ^ newName() 
		(* flat index value *)
            val ctr = "_ctr_" ^ newName() 
		(* loop counter (init to rank) *)
            val tmp = "_tmp_" ^ newName() 
		(* tmp register to check bounds *)
            val calc_name = "_calc_and_check_" ^ newName() 
		(* label name for loop *)
            val str_sz = "_strides_" ^newName() 
		(* Number of strides (to skip) *)

            (* Generates code for adding the indices to the stack *)
            fun copy_to_stack ([],code,n,l) = 
			   (Mips.ADDI(SP,SP, makeConst(~4*l)))::code 
              | copy_to_stack (i::inds,code,n,l) = 
                let 
                  val temp = "_temp_"^newName()
                in
                  copy_to_stack(inds, 
                    code @ (compileExp(vtab, i, temp)) @
                    [Mips.ADD (ind_reg, "0", temp),
                     Mips.SW (ind_reg, SP, makeConst n)], 
                  n+4,l)  
                end


            (* If basic type size is k, returns log2(k) 
	      (easier to calc later) *)
            fun element_size w = (case w of 
                (Array(a,Int)) => "2" 
              | (Array(a,Bool)) => "0" 
              | (Array(a,Char)) => "0"
              | _ => raise Error("Impossible!!!, ", pos))

            (* Initiates variables *)
            val init_code =
                [Mips.ADDI (arrptr, mem, "0"),
                 Mips.ADDI (ind_reg, "0", "0"),
                 Mips.ADDI (flat, "0", "0"),
                 Mips.ADDI (ctr, "0", makeConst rank),
                 Mips.ADDI (str_sz, "0", makeConst strides)]

            (* Checks if array index is out of bounds *)
            val calc_out_of_bounds =
                [Mips.SUB (tmp, dim_reg, ind_reg),
                 Mips.SLT (dim_reg, dim_reg, tmp),
                 Mips.SLTI (tmp, tmp, "1"),
                 Mips.OR (tmp, tmp, dim_reg),
                 Mips.BNE (tmp, "0", "_IllegalArrIndexError_")]

            (* Loop: checks if each index is within bounds 
	       and calculates flat index *)
            val calc_and_check =
                [Mips.LABEL (calc_name),
                 Mips.LW (dim_reg, arrptr, "0"),    
		(* loads current dim *)
                 Mips.MUL (flat, flat, dim_reg), 
                 Mips.LW (ind_reg, SP, "0")]        
		(* loads current index *)
                 @ calc_out_of_bounds @
                [Mips.ADD (flat, flat, ind_reg), 
                 Mips.ADDI (arrptr, arrptr, "4"),   
		(* points to next index *)
                 Mips.ADDI (SP, SP, "4"),           
		(* points to next dim *)
                 Mips.ADDI (ctr, ctr, "-1"),
                 Mips.BNE (ctr, "0", calc_name)]

            (* Calculates the address of the array index *)
            val get_address =
                [Mips.ADD(arrptr, arrptr, str_sz),      
		 (* skip the strides *)
                 Mips.LW(arrptr,arrptr,"0"),
                 Mips.SLL (flat, flat, element_size t),
                 Mips.ADD (arrptr, arrptr, flat)]
          in
            (copy_to_stack(inds,[],0,rank) @
             init_code @
             calc_and_check @
             get_address,
             Mem arrptr)
          end    

\end{lstlisting}
\newpage
\begin{lstlisting}
(*
                                   xx,
                                  xxxx;
xxxxxxxxxxxxxxxxxxxxxxxxxxxxx     xxx     xxxxxxxxxxxxxxxxxxxxxxxxxxxxx
 xxxxxxxxxxxxxxxxxxxxxxxxxxxx,   ;xxx;   .xxxxxxxxxxxxxxxxxxxxxxxxxxxx
  xxxxxxxxxxxxxxxxxxxxxxxxxxxx   xxxxx   xxxxxxxxxxxxxxxxxxxxxxxxxxxx
   `xxxxxxxxxxxxxxxxxxxxxxxxxxx..xxxxx..xxxxxxxxxxxxxxxxxxxxxxxxxxx`
     xxxxxxxxxxxxxxxxxxxxxxxxxxxxxxxxxxxxxxxxxxxxxxxxxxxxxxxxxxxxx
      xxxxxxxxxxxxxxxxxxxxxxxxxxxxxxxxxxxxxxxxxxxxxxxxxxxxxxxxxxx
       ^xxxxxxxxxxxxxxxxxxxxxxxxxxxxxxxxxxxxxxxxxxxxxxxxxxxxxxxx
        `xxxxxxxxxxxxxxxxxxxxxxxxxxxxxxxxxxxxxxxxxxxxxxxxxxxxx~
                         `xxxxxxxxxxxxxxxxxxxx
                          `xxxxxxxxxxxxxxxxx~
                            `xxxxxxxxxxxxx~
                                xxxxxxx
                                 xx xx
                                 x   x
*)
let val new_vtab = bindTypeIds(fargs, aargs, fid, pdcl, pcall)
    val res  = execBlock( body, new_vtab, ftab )
in  ( case (rtp, res) of
        (NONE , _) =>
        let 
            fun call_args [] [] = NONE
              | call_args (x::xs) (y::ys) =
                  let 
                    val hitler = updateOuterVtable vtab new_vtab (x, y)
                  in
                    call_args xs ys
                  end
              | call_args _ _ = raise Error("Number of functions args does not match declaration, at ", pdcl)
        in 
            call_args aexps fargs  
        end
\end{lstlisting}
\newpage
\begin{lstlisting}
and updateOuterVtable vtabOuter vtabInner (TpAbSyn.LValue (lval1, pos1), 
					   TpAbSyn.Dec ((id2,tp), pos2)) 
								      = 
     let val lenin = (case lval1 of 
                Var(id,tp) => id
              | Index((id,tp),e) => id)
     in
         case (SymTab.lookup lenin vtabOuter, 
	       SymTab.lookup id2 vtabInner) of
              (SOME x, SOME y) => x := !y
             | _               => raise Error("Procedure argument " ^
					      "not in caller", pos1)
     end
 | updateOuterVtable _ _ _ = ()

\end{lstlisting}
























\end{document}