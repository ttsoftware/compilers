\documentclass[12pt,a4paper,english]{article}
\usepackage [utf8]{inputenc}
\usepackage [english]{babel}
\usepackage [T1]{fontenc}
\usepackage {amsmath}
\usepackage{amsfonts}
\usepackage{amssymb}
\usepackage [top=3.4cm, bottom=2cm, left=3cm, right=3cm] {geometry}
\usepackage{setspace}
\usepackage{titlesec}
\usepackage{pgfkeys}
\titleformat{\section}[block]{\Large\bfseries\filcenter}{}{1em}{}
\titleformat{\subsection}[hang]{\bfseries\filcenter}{}{1em}{}
\setcounter{secnumdepth}{0}
\usepackage{pgfgantt}
\setcounter{tocdepth}{3}
\usepackage{graphicx}
\usepackage{geometry}
\usetikzlibrary{positioning,shapes, shadows, arrows}
\usepackage{tikz}
\usepackage{url}
\linespread{1.2}
\usepackage{fancyhdr}
\pagestyle{fancyplain}
\fancyhead[L]{Paladim Assignment}
\fancyhead[C]{}
\fancyhead[R]{December 20 2013}
\usepackage{listings}
\lstset{moredelim=[s][\color{gray}]{(*}{*)}}
\lstset{morecomment=[l][\color{gray}]{//}}
\lstset{moredelim=[s][\color{green!50!brown}]{"}{"}}
\lstset{moredelim=[s][\color{gray}]{/*}{*/}}
\lstset{emph={char,int,bool,string,array,Bool,Int,Char,Array,String},emphstyle=\color{yellow!50!brown},
	  emph={[2]NONE,SOME},emphstyle={[2]\color{orange}},
	  emph={[3]let,fun,map,return,val,in,case,of,end,and,var,begin,token,type,nonassoc,right,left,start,handle,if,then,else,andalso,orelse},emphstyle={[3]\color{brown}},
	  emph={[4]program,function,procedure,call_args,hitler,updateOuterVtable,lenin,evalExp,
	  evalOr,evalBinop,evalNot,print,read,chr,write, compileExp, pp_exp, typeOfExp, posOfExp,newName, Times, Div, Or, Not,pp_type,typeCheckExp,Mips},emphstyle={[4]\color{blue}},
	  emph={[5]},emphstyle={[5]\color{orange!50!red}}}


\begin{document}

\begin{titlepage}
    \vspace*{\fill}
    \begin{center}
      {\Huge Paladim Compiler}\\[0.7cm]
      {\large Alexander Worm Olsen - bdj816}\\[0.4cm]
      {\large Chi Dan Pham - vqr853}\\[0.4cm]
      {\large Troels Thompsen - qvw203}\\[0.4cm]
      {\small Group Project}\\[0.3cm]
      {\small December 20 2013}\\[0.3cm] 
      {\small Departmen of Computer Science}\\
      {\small University of Copenhagen}
    \end{center}
    \vspace*{\fill}
\end{titlepage}

\tableofcontents
\newpage
\section{Introduction}
The goal of the group assignment is to implement a compiler for the Paladim language using a bottom-up grammar. The langauge we use for this implementation is Stanard ML.
The majority of the compiler has already been implemented, and the changes we made are described in the individual tasks below. 
The first two tasks of the assigment was completed as a milestone assignment. We have made slight changes since the milestone, which improves the work done in these two tasks. A full description of these tasks and all the changes we made, are also included here.

\section{Task 1}
We were asked to implement the grammar production rules for the paladim language, which is described in the group project description document.

In order to implement the grammar, we have edited the empty file \textit{Parser.grm}, which was already included in the source code handout. We also edited the \textit{Driver.sml} and the \textit{Lexer.lex} source code file, to use our new parser structure instead of the LL1Parser which was included in the handout. \\

For inspirational purposes we used the file \textit{example.pdf} which we found on absalon, and the groupproject description document. In these documents we found examples for creating type precedence, how to construct terminals and non-terminals, and how to use them correctly in the grammar.  \\

The following sections describe the changes made to each file respectively.
\subsection{Parser.grm}
\subsubsection{Terminals}
The first definition in the parser, is the definition of terminals, which in mosmlyac is called tokens.
Here we use the abstract syntax type definitions to define the types of our tokens. The tokens themselves are defined in the \textit{Lexer.lex} file. We simply found all tokens in the lexer, and defined tokens for them.\footnote{See Appendix A. Parser.grm Listing 1.}

\subsubsection{Precedence Rules}

Next we define our precedence rules, to prevent shift / reduce conflicts. The rules at the bottom take the highest precedence.
We use left associative precedence for arithmetic operations, ranking \textit{TTimes} and \textit{TSlash} higher than \textit{TPlus} and \textit{TMinus}. \\

We define comparison operations \textit{TLess} and \textit{TEq} as non-associative, since it is not specified whether comparison is associative in Paladim in the project description. \\

The logical operation \textit{TNot} is defined as non-associative because it is unary. We define the other logical operations \textit{TAnd} and \textit{TOr} as left-associative (they can be either left- or right-associative, but should not be non-associative). Here \textit{TNot} takes precedence over \textit{TAnd} which takes precedence over \textit{TOr}. \\

The last two precedence rules are defined to resolve the shift / reduce conflict, which occurs when trying to write productions for If-Then-Else and If-Then. This conflict occurs in the following scenario. 
\begin{lstlisting}
if a > b then
    if c = d then
        return a
    else
        return b
\end{lstlisting}
In this scenario the parser does not know whether to match the production for If-Then-Else, or the production for If-Then. If it shifts, the else belongs to the inner if-statement, but if it reduces, the else belongs to the outer if-statement. In this scenario, we actually always wants to shift, because the else should belong to the closest previous if-then-statement. In order to accomplish this, we define TElse to take precedence over "LowPrec". We later assign "LowPrec" to the If-Then production grammar, and thus resolve the shift / reduce conflict.\footnote{See Appendix A. Parser.grm Listing 1.}

\subsubsection{Non-Terminals}

In this section we define our start symbol, and the non-terminals for our productions. \\
The types used for the non-terminals are defined in \textit{AbSyn.sml}. The actual non-terminals are defined in figure 3 in the group project definition document. The start symbol "Program" was defined by us, in order to handle end of file.\footnote{See Appendix A. Parser.grm, Listing 2}

\subsubsection{Grammar}

In this section we define the grammar\footnote{See Appendix A. Parser.grm Listing 2-5} used by the parser. For easier analysis we have divided the grammar into sections.
In general we built the grammar by looking at the production definitions in figure 3 in the group project definition. The type we return in the productions, are defined in the \textit{AbSyn.sml}. 

\subsubsection{Program structure}

The most important change in this section, is the Block production. To begin with we had a DBlock production, as described in figure 3 in the group project definition. This caused a shift / reduce conflict, which we resolved by combining the DBlock with the Block production.\footnote{See Appendix A. Parser.grm Listing 3.}

\subsubsection{Statements, Values and Expressions}
The most important change in this section, is the \textit{TIf Exp TThen Block \%prec LowPrec} production. As described in the non-terminal definitions, we need to assign lower precedence to the If-Then production. This is done by adding \%prec LowPrec at the end of the production.\footnote{See Appendix A. Parser.grm Listing 3.}

\textbf{The Exp LVal production} in this section might be problematic. This production needs to return an AbSyn.LValue(LVal, Pos). However this is not possible, because we call another production which only returns an LVal without a position. To solve this issue, we changed the non-terminal definition from <AbSyn.LVAL> to <AbSyn.LVAL*AbSyn.Pos>\footnote{See Appendix A.  Parser.grm Listing 2.}. This allows us to give our LValue the correct position, without modifying the syntax definition in AbSyn.sml.

\subsubsection{Parameters and Procedures}
At the end we have our simple definitions for parameters and declarations.\footnote{See Appendix A. Parser.grm Listing 5.}

\subsection{Driver.sml}
In the driver file we have uncommented the two lines which was meant for the LL1 parser and instead inserted the appropriate ones for our parser (line 57 and 69). Furthermore we have uncommented the Parser.ParseError, because this error message will be caught by the Parsing.ParseError, and instead only created compile errors.\footnote{See appendix B. Driver.sml Listing 6.}
\newpage
\subsection{Lexer.lex}
In the \textit{Lexer.lex} we have changed all instances of LL1parser to Parser, thereby including our newly created Parser.grm instead of the handout. Please note that this code is left out.

\section{Task 2}
In this task we were to implement the functionality of multiplication, division, or and not in Paladim. To do this we had to go through several source code files and implement the functionality of each of them. The files we've changed include; \textit{Compiler.sml},\textit{TpInterpret.sml},\textit{Type.sml} and \textit{TpAbSyn.sml}. Furthermore we were to make sure our precedence for boolean operators and arithmetic operations were correct, please note that this was done already in task 1\footnote{See Appendix A. Parser.grm Listing 1.}
\subsection{TpAbSyn.sml}
In order to implement the functionality of the four operations we started in \textit{TpAbSyn.sml}, where we first uncommented the four of them in the datatype for \textit{exp}, afterwards we made pretty printing for the four of the functions with inspiration from the ones already implemented, then a function that was able to find the position of each of the operations and at last we implemented the functionality to find the type of the operations.\footnote{See Appendix C. TpAbSyn.sml Listing 7.}
\subsection{TpInterpret.sml}
In this file we implemented the functionality of the four functions for the Interpreter, this was done with inspiration from the ones already implemented, this was done in \textit{evalExp}. In order to implement the to logical operations we had needed the a helper function for each of them. These changes made the functionality of our four functions work in the interpreter.\footnote{See Appendix D. TpInterpret.sml Listing 8.}
\subsection{Type.sml}
In order to implement the functions for our compiler aswell we had to include them in the \textit{type.sml}. These were aswell inspired by the ones already there. Please note that the types stated weren't added until task 3, and that the one for multiplication is left out because it's very similar to the one for division.\footnote{See Appendix E. Type.sml Listing 9}
\subsection{Compiler.sml}
The last step for the full implementation of multiplication, division, or and not were to implement them in \textit{compiler.sml}. The functionality of the four operations were straight forward and inspired by the ones already implemented. For the functionality of not in Mips code we used the mips instruction $XORI$. This is due to the fact that there doesn't exist a $NOT$ instruction in Mips. $XOR$ returns $1$ (true) if one of the arguments is different from $1$. In order to use this in the not scenario we pass the value to be negated, together with $1$ as the second argument.$XORI$ then returns true if the input argument is $0$ and false if the argument is $1$, as we would expect of not. 

\section{Task 3}
In this task we were to implement typechecking for Paladims library functions Read() and New()
\subsection{Type.sml}
The major change in the task, was changing the typeCheckExp function, to correctly typecheck the functions $read$ and $new$. In order to accomplish this, the typechecker needs to determine the expected type of the argument calling context. In the majority of the cases we changed from UnknownType to KnownType (<expected type>). \\

The actual changes can be found in Listing 9, 10 and 11 in appendix E. Two operations differ from this; Less and Equal, these two still holds UnkownType for argument1 but here the second argument is the same as what the 1. argument turns out to be. This is due to the fact that they are polymorphic. Even tho they are polymorphic, the types of both arguments needs to be equal. We cannot infer exactly what types, but only that they must be equal.

The major reason we do not inherit the expected type, is due to the fact that not all of our operators are polymorphic.\footnote{Appendix E. Type.sml Listing 10, 11 and 12.}

\section{Task 4}

\section{Task 5}
In this task we were to change the way Paladims procedures passes their arguments from call-by-value to call-by-value-result.
\subsection{TpInterpreter.sml}
In order to make Paladim procedures use call-by-value-result we change the \textit{TpInterpreter.sml}. We use the references from our innerVtable and swap them with the references for the according arguments in the outerVtable.\footnote{See appendix D. TpInterpreter.sml Listing 9} To do this we used the already implemented arguments \textit{fargs} and \textit{aexps}.

\subsection{Compiler.sml}
In order to implement call-by-value-result in the compiler, we created a function similar to putArgs called popArgs, which takes the same arguments as putArgs, plus the Mips codes returned by putArgs. \\
PopArgs then checks whether or not the expression is an LValue, and the Mips code is an ORI. In this case popArgs takes the caller register which is the second argument of ORI. PopArgs then stores the value in the temporary procedure argument register, in the caller register. \\
In any other case where the Mips command is not an ORI operation, or the expression is not an LValue expression, we simply continue the recursion. We do this in order to account for procedure arguments which are not variables, but constant values or constant expressions. \\

PopArgs is called right after putArgs in $compileStmt$, and simply appended to the existing mvcode\footnote{See appendix F. Compiler.sml Listing 17.}.\\

We also added a small change to $compileF$. The variable $new_argcode$ is used to store the Mips code created by popArgs, if we are compiling a procedure which is not main procedure. This code is then appended to the existing body of the procedure\footnote{See appendix F. Compiler.sml Listing 17.}. This is required in order for the register allocator to correctly swap the register values. 

\newpage
\section{Testing}
We have made two test examples; one that covers the problem with nested \textit{if then else} and another one that tests precedence with multiplication and addition. We ran the compiled programs in Mars and they returned the expected results ("If-then-else virker!!!" and "det virker jo").


\newpage

\section{Appendix A. Parser.grm}
\begin{lstlisting}[caption=The tokens and precedence rules for our grammar in Parser.grm.]
%{ 
   
%}

/* Token type definitions (will often be used in the Lexer)
 * Tokens use position attribute for demonstration 
 * (see below for Lexer)
 * As mentioned, the SML code above ends up after
 * this data declaration,
 * so we cannot use any types defined above
 * at this point of the file.
 * tokens needs to be of sml types
 * through the rules the should be converted to an absyn syntaxtree */

%token <AbSyn.Pos> TProgram TFunction TProcedure TVar TBegin TEnd 
		   TIf TThen TElse TWhile TDo TReturn TArray TOf 
		   TInt TBool TChar TAnd TOr TNot TAssign TPlus TMinus 
		   TTimes TSlash TEq TLess TLParen TRParen TLBracket 
		   TRBracket TLCurly TRCurly TComma TSemi TColon TEOF
%token <bool*AbSyn.Pos> TBLit
%token <int*AbSyn.Pos> TNLit
%token <char*AbSyn.Pos> TCLit
%token <string*AbSyn.Pos> TSLit TId

%nonassoc LowPrec
%nonassoc TElse
%right TOr
%right TAnd
%nonassoc TNot
%nonassoc TEq TLess
%left TPlus TMinus
%left TTimes TSlash

/* start symbol */
%start Program
\end{lstlisting}

\begin{lstlisting}[caption=The types and some of the grammar for the Parser.grm. (continued from listing 1)]
/* types returned by rules below */
%type <AbSyn.Prog> Program
%type <AbSyn.Prog> Prog
%type <AbSyn.Prog> FunDecs
%type <AbSyn.FunDec> FunDec
%type <AbSyn.StmtBlock> Block 
%type <AbSyn.Stmt list> SBlock
%type <AbSyn.Stmt list> StmtSeq
%type <AbSyn.Stmt> Stmt
%type <AbSyn.LVAL*AbSyn.Pos> LVal
%type <AbSyn.Exp option> Ret
%type <AbSyn.Exp> Exp
%type <AbSyn.Dec list> PDecl Params
%type <AbSyn.Dec> Dec
%type <AbSyn.Dec list> Decs
%type <AbSyn.Type> Type
%type <AbSyn.Exp list> CallParams Exps

%%

/* rules - a separate start rule is added automatically */

/* PROGRAM STRUCTURE*/
Program: Prog TEOF                 { $1 } 
;

Prog : 
    TProgram TId TSemi FunDecs     { $4 }
;

FunDecs :
    FunDecs FunDec                 { $1 @ [$2] }
  | FunDec                         { [$1] }
;
\end{lstlisting}
\newpage
\begin{lstlisting}[caption=More of the grammar for our Parser.grm. (continued from listing 2)]
FunDec :
    TFunction TId TLParen PDecl TRParen TColon Type Block TSemi
                                   { AbSyn.Func($7, #1 $2, $4, $8, $1) }
  | TProcedure TId TLParen PDecl TRParen Block TSemi
                                   { AbSyn.Proc(#1 $2, $4, $6, $1) }
;

Block :
    SBlock                         { AbSyn.Block ([], $1) }
  | TVar Decs SBlock               { AbSyn.Block ($2, $3) }
;

SBlock :
    TBegin StmtSeq TSemi TEnd      { $2 }
  | Stmt                           { [$1] }
;

StmtSeq :
    StmtSeq TSemi Stmt             { $1 @ [$3] }
  | Stmt                           { [$1] }
;

/* STATEMENTS */
Stmt :
    TId TLParen CallParams TRParen { AbSyn.ProcCall (#1 $1, $3, #2 $1) }
  | TIf Exp TThen Block TElse Block{ AbSyn.IfThEl ($2, $4, $6, $1) }
  | TIf Exp TThen Block %prec LowPrec 
			           { AbSyn.IfThEl ($2, $4, 
						   AbSyn.Block([],[]), 
					    	   $1)}
  /* prec gives precedence as LowPrec */
  | TWhile Exp TDo Block           { AbSyn.While ($2, $4, $1) }
  | TReturn Ret                    { AbSyn.Return ($2, $1) }
  | LVal TAssign Exp               { AbSyn.Assign ($1, $3, $2) }
;
\end{lstlisting}
\newpage
\begin{lstlisting}[caption=More of the grammar for our Parser.grm. (continued from listing 3)]
/* L-VALUES AND EXPRESSIONS */
LVal :
    TId                            { AbSyn.Var (#1 $1) }
  | TId TLBracket Exps TRBracket   { AbSyn.Index (#1 $1, $3) }
;

Ret :
    Exp                            { SOME $1 }
  |                                { NONE }
;

Exp :
    TNLit                          { AbSyn.Literal (AbSyn.BVal(
				     AbSyn.Num(#1 $1)), #2 $1) }
  | TBLit                          { AbSyn.Literal (AbSyn.BVal(
				     AbSyn.Log(#1 $1)), #2 $1) }
  | TCLit                          { AbSyn.Literal (AbSyn.BVal(
				     AbSyn.Chr(#1 $1)), #2 $1) } 
  | TSLit                          { AbSyn.StrLit (#1 $1, #2 $1) }
  | TLCurly Exps TRCurly           { AbSyn.ArrLit ($2,$1) }
  | LVal                           { AbSyn.LValue (#1 $1, #2 $1) }
  | TNot Exp                       { AbSyn.Not ($2, $1) }
  | Exp TPlus Exp                  { AbSyn.Plus ($1, $3, $2) }
  | Exp TMinus Exp                 { AbSyn.Minus ($1, $3, $2) }
  | Exp TTimes Exp                 { AbSyn.Times ($1, $3, $2) }
  | Exp TSlash Exp                 { AbSyn.Div ($1, $3, $2) }
  | Exp TEq Exp                    { AbSyn.Equal ($1, $3, $2) }
  | Exp TLess Exp                  { AbSyn.Less ($1, $3, $2) }
  | Exp TAnd Exp                   { AbSyn.And ($1, $3, $2) }
  | Exp TOr Exp                    { AbSyn.Or ($1, $3, $2) }
  | TLParen Exp TRParen            { $2 }
  | TId TLParen CallParams TRParen { AbSyn.FunApp (#1 $1, $3, # 2$1) }
;

/* VARIABLE AND PARAMETER DECLARATIONS, TYPES */
PDecl : 
    Params                         { $1 }
  |                                { [] } ;
\end{lstlisting}

\begin{lstlisting}[caption=The last of the grammar for our Parser.grm. (continued from listing 4)]
Params :
    Params TSemi Dec               { $1 @ [$3] }
  | Dec                            { [$1] }
;

Dec :
    TId TColon Type                { AbSyn.Dec (#1 $1, $3, #2 $1) }
;

Decs :
    Decs Dec TSemi                 { $1 @ [$2] }
  | Dec TSemi                      { [$1] }
;

Type :
    TInt                           { AbSyn.Int ($1) }
  | TChar                          { AbSyn.Char ($1) }
  | TBool                          { AbSyn.Bool ($1) }
  | TArray TOf Type                { AbSyn.Array ($3,$1) }
;

/* FUNCTION AND PROCEDURE PARAMETERS AND INDEX LISTS */
CallParams :
    Exps                           { $1 }
  |                                { [] }
;

Exps :
    Exp TComma Exps                { $1 :: $3 }
  | Exp                            { [$1] }

%%

(* SML trailer *)
\end{lstlisting}

\newpage
\section{Appendix B. Driver.sml}

\begin{lstlisting}[caption=The Driver.sml changes for our parser.grm.]
fun compile arg path =
  let
    val inpath = path
    val outpath= Path.base path ^ ".asm"
    val lexbuf = createLexerStream (BasicIO.open_in inpath)
  in
    let
      (*val pgm = LL1Parser.parse Lexer.Token lexbuf*)
      (* COMMENT LINE ABOVE AND UNCOMMENT  *)
      (* THE LINE BELOW TO USE YOUR PARSER *)
      val pgm = Parser.Program Lexer.Token lexbuf
    in case arg of
      "-ti" => typedInterpret (typeCheck pgm)
    | "-c"  => compileNormal pgm outpath
    | other => print ("'" ^ other ^ "': Unknown mode of operation.\n")
    end
    handle
      Parsing.yyexit ob => errorMsg "Parser-exit\n"
    | Parsing.ParseError ob =>
       (* errorMsgAt "Parsing error" (Lexer.getPos lexbuf) *)
       (* COMMENT LINE ABOVE AND UNCOMMENT  *)
       (* THE LINE BELOW TO USE YOUR PARSER *)
         errorMsgAt "Parsing error" (Lexer.getPos lexbuf)

   (* | Parser.ParseError s =>
         errorMsgAt ("Parse error: " ^ s) (Lexer.getPos lexbuf) *)
\end{lstlisting}

\newpage
\section{Appendix C. TpAbSyn.sml}

\begin{lstlisting}[caption=TpAbSyn.sml with the functionality of the four operations implemented for task 2.]
| Times   of Exp * Exp         * Pos      (* e.g., x * 3 *)
| Div     of Exp * Exp         * Pos      (* e.g., x / 3 *) 
| Or      of Exp * Exp         * Pos (* e.g., (x=5) or y *)
| Not     of Exp               * Pos (* e.g., not (x>3)  *)

| pp_exp (Or    (e1, e2, _))    = "( " ^ pp_exp e1 ^ " | " ^ 
				  pp_exp e2 ^ " )"
| pp_exp (Not   (e1,     _))    = "( not " ^ pp_exp e1 ^ " )"
| pp_exp (Times (e1, e2, _))    = "( " ^ pp_exp e1 ^ " * " ^ 
				  pp_exp e2 ^ " )"
| pp_exp (Div   (e1, e2, _))    = "( " ^ pp_exp e1 ^ " / " ^
				  pp_exp e2 ^ " )"

| typeOfExp ( Or     (_,_,_) ) = BType Bool
| typeOfExp ( Not    (_,  _) ) = BType Bool
| typeOfExp ( Times  (a,b,_) ) = typeOfExp a
| typeOfExp ( Div    (a,b,_) ) = typeOfExp a

| posOfExp  ( Or     (_,_,p) ) = p
| posOfExp  ( Not    (_,  p) ) = p
| posOfExp  ( Times  (_,_,p) ) = p
| posOfExp  ( Div    (_,_,p) ) = p
\end{lstlisting}

\newpage
\section{Appendix D. TpInterpret.sml}

\begin{lstlisting}[caption=TpInterpret.sml changes for task 2.]
fun evalOr  (BVal (Log b1), BVal (Log b2), pos) = 
	     BVal (Log (b1 orelse b2))
  | evalOr  (v1, v2, pos) =
        raise Error( "Or: argument types do not match. Arg1: " ^
                      pp_val v1 ^ ", arg2: " ^ pp_val v2, pos )
                      
fun evalNot  (BVal (Log b1), pos) = BVal (Log (not b1))
  | evalNot  (v1, pos) =
        raise Error( "Not: argument types do not match. Arg1: " ^
                      pp_val v1, pos )

  | evalExp ( Div(e1, e2, pos), vtab, ftab ) =
        let val res1   = evalExp(e1, vtab, ftab)
            val res2   = evalExp(e2, vtab, ftab)
        in  evalBinop(op div, res1, res2, pos)
        end

  | evalExp ( Times(e1, e2, pos), vtab, ftab ) =
        let val res1   = evalExp(e1, vtab, ftab)
            val res2   = evalExp(e2, vtab, ftab)
        in  evalBinop(op *, res1, res2, pos)
        end

  | evalExp ( Or(e1, e2, pos), vtab, ftab ) =
        let val r1 = evalExp(e1, vtab, ftab)
            val r2 = evalExp(e2, vtab, ftab)
  	in  evalOr(r1, r2, pos)
  	end
    
  | evalExp ( Not(e1, pos), vtab, ftab ) =
        let val r1 = evalExp(e1, vtab, ftab)
        in  evalNot(r1, pos)
        end
\end{lstlisting}
\newpage
\begin{lstlisting}[caption=TpInterpreter changes for task 5.]
let val new_vtab = bindTypeIds(fargs, aargs, fid, pdcl, pcall)
    val res  = execBlock( body, new_vtab, ftab )
in  ( case (rtp, res) of
        (NONE , _) =>
        let 
            fun call_args [] [] = NONE
              | call_args (x::xs) (y::ys) =
                  let 
                    val hitler = updateOuterVtable vtab new_vtab (x, y)
                  in
                    call_args xs ys
                  end
              | call_args _ _ = raise Error("Number of functions args" ^
					    " does not match " ^ 
					    "declaration, at ", pdcl)
        in 
            call_args aexps fargs  
        end

and updateOuterVtable vtabOuter vtabInner (TpAbSyn.LValue (lval1, pos1), 
					   TpAbSyn.Dec ((id2,tp), pos2)) 
								      = 
     let val lenin = (case lval1 of 
                Var(id,tp) => id
              | Index((id,tp),e) => id)
     in
         case (SymTab.lookup lenin vtabOuter, 
	       SymTab.lookup id2 vtabInner) of
              (SOME x, SOME y) => x := !y
             | _               => raise Error("Procedure argument " ^
					      "not in caller", pos1)
     end
 | updateOuterVtable _ _ _ = ()

\end{lstlisting}

\newpage
\section{Appendix E. Type.sml}

\begin{lstlisting}[caption=Changes made in type.sml for task 2 and task 3.]
(* Must be modified to complete task 3 *)
| typeCheckExp( vtab, AbSyn.Div (e1, e2, pos), _ ) =
   let val e1_new = typeCheckExp(vtab, e1, KnownType (BType Int) )
        val e2_new = typeCheckExp(vtab, e2, KnownType (BType Int) )
        val (tp1, tp2) = (typeOfExp e1_new, typeOfExp e2_new)
   in  if  typesEqual(BType Int, tp1) andalso 
            typesEqual(BType Int, tp2)
        then Div(e1_new, e2_new, pos)
        else raise Error("in type check minus exp, one argument " ^
			 "is not of int type "^ pp_type tp1^
			 " and "^pp_type tp2^" at ", pos)
   end

(* Must be modified to complete task 3 *)
| typeCheckExp ( vtab, AbSyn.Or (e1, e2, pos), _ ) =
    let val e1_new = typeCheckExp(vtab, e1, KnownType (BType Bool) )
        val e2_new = typeCheckExp(vtab, e2, KnownType (BType Bool) )
        val (tp1, tp2) = (typeOfExp e1_new, typeOfExp e2_new)
    in  if  typesEqual(BType Bool, tp1) andalso 
	    typesEqual(BType Bool, tp2)
        then Or(e1_new, e2_new, pos)
        else raise Error("in type check and exp, one argument is " ^ 
			 "not of bool type "^
                           pp_type tp1^" and "^pp_type tp2^" at ", pos)
    end    

(* Must be modified to complete task 3 *)
| typeCheckExp ( vtab, AbSyn.Not (e1, pos), _ ) =
    let val e1_new = typeCheckExp(vtab, e1, KnownType (BType Bool) )
        val (tp1) = (typeOfExp e1_new)
    in  if  typesEqual(BType Bool, tp1)
        then Not(e1_new, pos)
        else raise Error("in type check and exp, one argument is not " ^
			 "of bool type " ^
                            pp_type tp1^" at ", pos)
    end   
\end{lstlisting}

\begin{lstlisting}[caption=Changes in type.sml for task 3. (continued from listing10)]
| typeCheckExp ( vtab, AbSyn.Less (e1, e2, pos), _ ) =
    let val e1_new = typeCheckExp(vtab, e1, UnknownType )
        val e2_new = typeCheckExp(vtab, e2, KnownType (typeOfExp e1_new) )
        val (tp1, tp2) = (typeOfExp e1_new, typeOfExp e2_new)
        (* check that tp1 is not an array type *)
        val () = case tp1 of
                   Array _ => raise Error("in type check less, " ^
					  "first expression "^ 
					  pp_exp e1_new ^
                                          "is an array (of type) " ^
					  pp_type tp1^" at ", pos)
                       | _ => ()
    in  if  typesEqual(tp1, tp2)
        then Less(e1_new, e2_new, pos)
        else raise Error("in type check less exp, argument types " ^
			 " do not match "^
                         pp_type tp1^" <> "^pp_type tp2^" at ", pos)
    end

| typeCheckExp ( vtab, AbSyn.Equal(e1, e2, pos), _ ) =
  let val e1_new = typeCheckExp(vtab, e1, UnknownType)
      val e2_new = typeCheckExp(vtab, e2, 
		   		KnownType (typeOfExp e1_new) )
      val (tp1, tp2) = (typeOfExp e1_new, typeOfExp e2_new)
      (* check that tp1 is not an array type *)
      val () = case tp1 of
               Array _ => raise Error("in type check equal, " ^
				      "first expression " ^
				      pp_exp e1_new ^
                                      "is an array (of type) " ^
				      pp_type tp1 ^ " at ", pos)
                   | _ => ()
  in  if  typesEqual(tp1, tp2)
      then Equal(e1_new, e2_new, pos)
      else raise Error("in type check equal exp, " ^
		       " argument types do not match " ^
                       pp_type tp1^" <> "^pp_type tp2^" at ", pos)
  end

\end{lstlisting}

\begin{lstlisting}[caption=changes in type.sml for task 3 and 4.]
(* function call to `new' uses expected type to infer
   the to-be-read result *)
| typeCheckExp ( vtab, AbSyn.FunApp ("new", args, pos), etp ) =
    ( case expectedBasicType etp of
        SOME btp => 
          let 
            val typedargs = map (fn n => 
		      typeCheckExp(vtab, n,KnownType (BType Int))) args
            val types = map typeOfExp typedargs
            val rtp = Array ( length args, btp)
          in
            if List.all (fn n => typesEqual(BType Int, n)) types
            then 
              FunApp(("new", (types, SOME rtp)), typedargs,pos)
            else 
              raise Error("declared array dimensions are not " ^
			  "integers, at ", pos)
          end

| typeCheckExp( vtab, AbSyn.LValue( AbSyn.Index(id, inds), pos ), _ ) =
    let
       val indices = length inds
       val newinds = map (fn n=>
			typeCheckExp(vtab,n,KnownType (BType Int))) inds
       val correctinds = List.all (fn n => 
			typesEqual(BType Int, typeOfExp n)) newinds
    in
       case SymTab.lookup id vtab of
          SOME (Array(r,t)) => 
             if indices > 0 andalso r = length inds andalso correctinds 
             then
               LValue(Index((id,Array(r,t)),newinds),pos)
             else
               raise Error("ill-formed array indexing at ",pos)
        | _ => raise Error("in type check variable, var " 
			   ^id^"not in VTab, at",pos)
    end
\end{lstlisting}

\newpage
\section{Appendix F. Compiler.sml}

\begin{lstlisting}[caption=The compiler.sml changes for task 2.]
| compileExp( vtable, Times (e1, e2, _), place ) =
    let val t1 = "times1_" ^ newName()
        val c1 = compileExp(vtable, e1, t1)
        val t2 = "times2_" ^ newName()
        val c2 = compileExp(vtable, e2, t2)
    in c1 @ c2 @ [Mips.MUL (place,t1,t2)]
    end

| compileExp( vtable, Div (e1, e2, _), place ) =
    let val t1 = "div1_" ^ newName()
        val c1 = compileExp(vtable, e1, t1)
        val t2 = "div2_" ^ newName()
        val c2 = compileExp(vtable, e2, t2)
    in c1 @ c2 @ [Mips.DIV (place,t1,t2)]
    end

| compileExp( vtable, Or(e1, e2, _), place ) =
    let val t1 = "or1_" ^ newName()
        val c1 = compileExp(vtable, e1, t1)
        val t2 = "or2_" ^ newName()
        val c2 = compileExp(vtable, e2, t2)
        val lA = "_or_" ^ newName()
    in 
        c1 @ c2 @ [Mips.OR (place,t1,t2)]
    end
        
| compileExp( vtable, Not(e1, _), place ) =
    let val t1 = "not1_" ^ newName()
        val c1 = compileExp(vtable, e1, t1)
        val lA = "_not_" ^ newName()
    in 
        c1 @ [Mips.XORI (place,t1,"1")]
    end
\end{lstlisting}
\newpage
\begin{lstlisting}[caption=Compiler.sml for Task 4.]
| compileLVal( vtab : VTab, Index ((n,t),inds) : LVAL, pos : Pos ) =
    ( case SymTab.lookup n vtab of
        SOME mem => 
          let
            val rank = length inds
            val strides = 4 * (rank - 1)
                
            (* Variables for generated MIPS code *)
            val arrptr = "_arrptr_" ^ newName() 
		(* pointer to a *)
            val ind_reg = "_indx_" ^ newName() 
		(* current index i_k *)
            val dim_reg = "_dimx_" ^ newName() 
		(* current dimension d_k *)
            val flat = "_flatIndx_" ^ newName() 
		(* flat index value *)
            val ctr = "_ctr_" ^ newName() 
		(* loop counter (init to rank) *)
            val tmp = "_tmp_" ^ newName() 
		(* tmp register to check bounds *)
            val calc_name = "_calc_and_check_" ^ newName() 
		(* label name for loop *)
            val str_sz = "_strides_" ^newName() 
		(* Number of strides (to skip) *)

            (* Generates code for adding the indices to the stack *)
            fun copy_to_stack ([],code,n,l) = 
			   (Mips.ADDI(SP,SP, makeConst(~4*l)))::code 
              | copy_to_stack (i::inds,code,n,l) = 
                let 
                  val temp = "_temp_"^newName()
                in
                  copy_to_stack(inds, 
                    code @ (compileExp(vtab, i, temp)) @
                    [Mips.ADD (ind_reg, "0", temp),
                     Mips.SW (ind_reg, SP, makeConst n)], 
                  n+4,l)  
                end
\end{lstlisting}

\begin{lstlisting}[caption=Compiler.sml task 4. (continued from listing 14)]
            (* If basic type size is k, returns log2(k) 
	       (easier to calc later) *)
            fun element_size w = (case w of 
                (Array(a,Int)) => "2" 
              | (Array(a,Bool)) => "0" 
              | (Array(a,Char)) => "0"
              | _ => raise Error("Impossible!!!, ", pos))
	    (* Initiates variables *)
	    val init_code =
    	        [Mips.ADDI (arrptr, mem, "0"),
     	         Mips.ADDI (ind_reg, "0", "0"),
     	         Mips.ADDI (flat, "0", "0"),
     	         Mips.ADDI (ctr, "0", makeConst rank),
     	         Mips.ADDI (str_sz, "0", makeConst strides)]

	    (* Checks if array index is out of bounds *)
	    val calc_out_of_bounds =
    	        [Mips.SUB (tmp, dim_reg, ind_reg),
     	         Mips.SLT (dim_reg, dim_reg, tmp),
     	         Mips.SLTI (tmp, tmp, "1"),
     	         Mips.OR (tmp, tmp, dim_reg),
     	         Mips.BNE (tmp, "0", "_IllegalArrIndexError_")]
\end{lstlisting}
\newpage
\begin{lstlisting}[caption=Compiler.sml task 4. (continued from listing 15)]
	    (* Loop: checks if each index is within bounds 
    	        and calculates flat index *)
	    val calc_and_check =
    		[Mips.LABEL (calc_name),
     		 Mips.LW (dim_reg, arrptr, "0"),    
		     (* loads current dim *)
     		 Mips.MUL (flat, flat, dim_reg), 
     	 	 Mips.LW (ind_reg, SP, "0")]        
		     (* loads current index *)
     		@ calc_out_of_bounds @
    		[Mips.ADD (flat, flat, ind_reg), 
     		 Mips.ADDI (arrptr, arrptr, "4"),   
		     (* points to next index *)
     		 Mips.ADDI (SP, SP, "4"),           
		     (* points to next dim *)
     		 Mips.ADDI (ctr, ctr, "-1"),
     		 Mips.BNE (ctr, "0", calc_name)]

	    (* Calculates the address of the array index *)
  	    val get_address =
      		[Mips.ADD(arrptr, arrptr, str_sz),      
  	  	      (* skip the strides *)
       		 Mips.LW(arrptr,arrptr,"0"),
       		 Mips.SLL (flat, flat, element_size t),
       		 Mips.ADD (arrptr, arrptr, flat)]
	in
           (copy_to_stack(inds,[],0,rank) @
           init_code @
           calc_and_check @
           get_address,
           Mem arrptr)
         end    
\end{lstlisting}
\newpage
\begin{lstlisting}[caption=Compiler.sml task 5]
(* Swap temporary registers with caller registers, after
   the procedure has finished *)
and popArgs (TpAbSyn.LValue(lval, pos)::es) vtable 
             reg ((Mips.ORI(rd, rs, v))::ts) =
      let 
          val code = popArgs es vtable (reg+1) ts
      in  
          code @ [Mips.MOVE (rs, makeConst reg)]
      end
  | popArgs (e::es) vtable reg (t::ts) = popArgs (e::es) vtable reg ts 
(* if expression has more than 1 mips command, we want to pass the same
 expression along again. *)
  | popArgs _ vtable reg [] = []
  | popArgs [] vtable reg _ = []

| ProcCall ((n,_), es, p) => 
  let
      val (mvcode, maxreg) = putArgs es vtable minReg
      val prod_codes = popArgs es vtable minReg mvcode
      val new_mvcode = mvcode
          @ [Mips.JAL (n, List.tabulate (maxreg, fn reg => makeConst reg))]
          @ prod_codes
  in
      new_mvcode
  end

val new_argcode = 
    if isProc andalso (not (fname = "main")) 
    then map (fn (vname, reg) => Mips.MOVE (reg, vname)) movePairs
    else []
val body = compileStmts block vtable (fname ^ "_exit")
val (body1, _, maxr, spilled) =  (* call register allocator *)
   RegAlloc.registerAlloc ( argcode @ body @ new_argcode )
\end{lstlisting}

\newpage

\section{Appendix G. Testing}
\begin{lstlisting}[caption=Two tests for task 1 one for dangling else and one for general functionality.]
// Test for dangling else
program dangling;

procedure main()
var i : int;
begin
   i := 0;
   if false then if true then i := 1 else i := 2;
   if i = 0 then write("passed.\n") else write("failed.\n");
end;

// General test for task 1
program test2;

function test() : bool
var m : int;
    o : int;
    v : int;
    p : int;
begin
    m := 2;
    o := 3;
    v := (4+m)*6100;
    p := 1+1*o*2;
    return v = 36600 and p = 7;
end;

procedure main()
begin
    if test() then 
        write("It works") 
    else 
        write("fail");
end;
\end{lstlisting}

\newpage
\begin{lstlisting}[caption=tests for task 2]
program precedence;

procedure main()
var 
    a : int;
    b : int;
    c : int;
    d : int;
    x : bool;
    y : bool;
    z : bool;
    q : bool;
begin
    a := 10;
    b := 5;
    c := 2;
    x := true;
    y := false;
    z := false;
    
    q := y = z or not x = y and x = z;
    write(q = true);
    
    q := x = z or not x = y and x = z;
    write(q = false);
    
    d := a + b * b - a / c;
    write(d = 30); 
    
end;
\end{lstlisting}

\newpage
\begin{lstlisting}[caption=tests for task 3]
program task3readtest;

procedure main()
var 
    x : int;
    y : bool;
begin
    // The first input should be an integer, hence chr() 
    // and the second input a char.
    // for example 1. input: 100, 2. input: d, output = 1 (true)
    y := chr(read()) = read();
    write(y);
    // the input for read should be of type integer
    x := 3 + read();
    write(x);
    // the input for read should be of type integer
    y := 3 = read();
    write(y);
    // the input for read should be a char
    y := 'a' = read();
    write(y);
    // the input for read should be an integer
    y := 3 < read();
    write(y);
    // the input for read should be of type bool
    y := false or read();
    write(y);
    // the input for read should be of type bool
    y := true and read();
    write(y);
    // the input for read should be of type bool
    y := not read();
    write(y);
end;
\end{lstlisting}

\newpage
\begin{lstlisting}[caption=tests for task 4]
program Inference;

procedure main()
var a : array of int;
    b : array of bool;
    c : array of char;
    d : array of array of array of int;
    e : array of array of array of array of array of array of int;
    v : bool;
begin
   a := {8, 0, 0, 8, 1, 3, 5};
   b := {true, true, true, false, false, true, true, true};
   c := {'w','o','w','s','u','c','h','a','r','r','a','y'};
   d := {
         {
           {1, 2}, {3, 4}
         },
         {
           {5, 6}, {7, 8}
         }
        };
   e := new(4,6,7,4,6,9);
   a[4] := 2;
   b[3] := true;
   c[1] := 'O';
   d[1, 1, 1] := 42;
   d[0, 0, 0] := a[0];
   e[2,2,2,2,2,2] := 5;
   write(e[2,2,2,2,2,2] = 5);
   write(d[0,0,0] = 8);
   write(a[4] = 2);
   e[2,2,2,2,2,2] := d[1,1,1];
   write(e[2,2,2,2,2,2] = 42);
   write(d[1,0,1] = 6);
   
end;
\end{lstlisting}

\newpage
\begin{lstlisting}[caption=tests for task 5 (please note this spands two pages)]
program proctest;

procedure f(a : int; 
            b : int)
begin
    a := a + 1;
    b := b + 4;
end;

procedure g(a : int;
            b : int)
begin
    a := a + 2;
    b := b + 3;
end;

procedure h(a : int;
            b : int)
begin
    f(a, b);
    a := a + 1;
    b := b + 1;
end;

procedure j(a : array of array of int;
            b : array of array of int)
begin
    a[0, 0] := 2;
    b[0, 0] := 3;
end;

function k(a : int;
           b : int) : int
begin
    a := a + 10;
    b := b + 5;
    return b;
end;

procedure main()
var
    x : int;
    y : int;
    z : array of array of int;
    q : array of array of int;
    r : int;
begin
   x := 2;
   y := 3;
   f(x, y);
   write(x = 3);
   write(y = 7);
   g(x, y);
   write(x = 5);
   write(y = 10);
   h(x, y);
   write(x = 7);
   write(y = 15);

   z := new(2, 2);
   q := new(2, 2);

   z[0, 0] := 1;
   q[0, 0] := 4;

   j(z, q);

   write(z[0,0] = 2);
   write(q[0,0] = 3);

   r := k(z[0,0], q[0,0]);

   write(r = 8);
   write(z[0,0] = 2);

   f(1, 1);
end;
\end{lstlisting}

\begin{lstlisting}[caption=Second test for procedure calls in task 5 including constants]
program proctest2;

procedure f(d : int; e : int; f : int;)
begin
    d := 3;
    e := 4;
    f := 5;
end;

procedure main()
var
    s : int;
    t : int;
    u : int;
begin
   s := 1;
   t := 1;
   u := 1;
   
   f(x,y,r,s,3+4,u,v,z,w);

   write(s = 3);
   write(u = 5);

end;


\end{lstlisting}





\end{document}